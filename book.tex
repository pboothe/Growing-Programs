\documentclass{book}

\usepackage{hyperref}
\title{Growing Programs: A Constructive Approach to Computational Problem Solving}
\author{Peter Boothe and John Lasseter}

\begin{document}
\maketitle

The big cross-cutting concern we have is growing programs --- in every
instance, we want to show how we are writing a program the wrong way, fixing
it, and then writing an equivalent program the right way.

Exploratory programming.

The ability to use and make patterns.

The ability to simplify problems.
The ability to add complexity appropriately.

The ability to generalize problems.
The ability to specialize problems.

In all cases, we want to emphasize that solutions should be grown.  Because
students don't believe what they read or are told, we must, with every
solution, show all of its growth stages.

\tableofcontents
\chapter{Printing things out and reading things in}
\section{How to grow an interactive program}

\chapter{Data types --- Strings and Ints and Floats and Booleans}
\section{How to grow a numerical program}

\chapter{Numbers and Math}
\section{How to grow a numerical program}
\section{The math library}

\chapter{Conditional Execution}
\section{How to grow a decision tree}

\chapter{Loops}
\section{How to grow a for loop}
\section{How to grow a while loop}

\chapter{Lists and Dictionaries}
\section{How to grow a program using lists}
\section{How to grow a dictionary-using program}

\chapter{Functions}
\section{How to grow a function}
\section{How to grow a recursive function}
\section{Structured design}
\section{How to grow a structured design}

\chapter{Objects}
\section{How to grow a program using objects}

\chapter{Classes}
\section{How to grow a class}
\end{document}
